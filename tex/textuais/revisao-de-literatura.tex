\chapter{REVISÃO DE LITERATURA}\label{chap:fundamentacaoTeorica}

Fazer referências às fontes consultadas:
\\

\\
\textbf{Manuais de Produto}:

Especificação do Produto~\cite{mpu6050ps},

Descrição e Mapa de Registradores~\cite{mpu6050rm}.
\\

\\
\textbf{Controle e Robótica}:

Sistemas de Controle Moderno~\cite{Ogata2010},

Introdução à robótica~\cite{Craig2014},
\\

\\
\textbf{Ângulos de Euler}:

Ângulos de Euler~\cite{Henderson1997},
\\

\\
\textbf{Sistemas de Navegação Inercial}:

Navegação Inercial Básica~\cite{Stovall1997}.

Navegação Inercial Fixada no Corpo~\cite{Weston2004},

Sistemas de Controle de Aeronaves~\cite{Stevens2016},

Controle Automático de Aeronaves e Mísseis~\cite{Blakelock1991},

Navegação Inercial Pedestre~\cite{Wang2021},

Sistemas de Navegação Embarcados~\cite{Haoran2019}

\section{Notação e convenções}

Neste trabalho utilizamos um modelo de mundo tridimensional e mecânica
clássica. Em outras palavras, nosso espaço é estruturado por três eixos
ortogonais e descrevemos a posição com um vetor tridimensional. Desse modo,
descrevemos a atitude e o movimento de um corpo rígido sobre a superfície
oblata e girante da Terra. Não obstante, serão empregadas aproximações sobre
uma pequena área para uma terra plana, estacionária e constante, que é acurado
o bastante para os nossos propósitos.

Utilizamos as mesmas convenções em Stevens~\cite{Stevens2016}:

\begin{align*}
    \mathbf{p}_{A/B} &\equiv
    \text{ vetor posição do ponto A em relação ao ponto B} \\
    \mathbf{v}_{A/i} &\equiv
    \text{ vetor velocidade do ponto \(A\) no quadro \(F_{i}\)} \\
    ^{b}\mathbf{\dot{v}}_{A/i} &\equiv
    \text{ vetor derivada de \(\mathbf{v}_{A/i}\) tomada no quadro \( F_{b}\)} \\
    \mathbf{v}^{c}_{A/i} &\equiv \left( \mathbf{v}_{A/i} \right)^{c} \equiv
    \text{ conjunto dos componentes de \(\mathbf{v}_{A/i}\) no sistema de coordenadas \(c\)} \\
\end{align*}

Componentes de vetores terão subscritos para indicar o sistema de coordenadas ou serão denotados pelo símbolo de vetor com subscritos \(x\), \(y\), e \(z\) para indicar as coordenadas, exceto quando indicado pelo símbolo transposto, e todos vetores são do tipo vetor coluna. Por exemplo:

\begin{align*}
    &\mathbf{p}^{b}_{A/B} = \begin{bmatrix}x_{b} \\ y_{b} \\ z_{b}\end{bmatrix}&
    \text{ or }&
    &\mathbf{v}^{b}_{A/i} = \begin{bmatrix} v_{x} \\ v_{y} \\ v_{z} \end{bmatrix} = \begin{bmatrix} v_{x} & v_{y} & v_{z} \end{bmatrix}^{T}
\end{align*}

\section{Descrevendo a atitude de um corpo}

\textbf{ADAPTAR de \citeonline{Stevens2016}.}

Standard aircraft practice is to describe aircraft orientation by
the \(z\), \(y\), \(x\) (also called 3, 2, 1) right-handed Euler
rotation sequence that is required to get from a reference system
on the surface of Earth into alignment with an aircraft body-fixed
coordinate system. The usual choice for the reference system, on
Earth, is a North-East-down (\(ned\)) system, with the \(x\)-axis
pointing true North, the z-axis pointing down, and the \(y\)-axis
completing the right-handed set. The exact meaning of ``down'' will
be explained in Section 1.6 The aircraft axes are normally aligned
(\(x\),\(y\),\(z\)), forward, right, and down (\(frd\)), with
``forward'' aligned with the \textit{longitudinal reference line}
of the aircraft, and the forward and down axes in the aircraft
plane of symmetry. Therefore, starting from the reference system,
the sequence of rotations is:

\begin{enumerate}
    \item Right-handed rotation about the \(z\)-axis, or positive \(\psi\) (compass heading)
    \item Right-handed rotation about the new \(y\)-axis, or positive \(\theta\) (pitch)
    \item Right-handed rotation about the new \(x\)-axis, or positive \(\phi\) (roll)
\end{enumerate}

The rotations are often described as yaw-pitch-roll sequence,
starting from the reference system.

The plane rotation matrices can be written down immediately with
the help of the rules estabilished in the preceding subsection.
Thus, abbreviating cosine and sine to \(c\) and \(s\), we have,

\begin{align*}
    C_{f\!r\!d\!/\!n\!e\!d} =
    \begin{bmatrix}
        1               &  0            &  0             \\
        0               &  \cos{\phi}   &  \sin{\phi}    \\
        0               & -\sin{\phi}   &  \cos{\phi}
    \end{bmatrix}
    \begin{bmatrix}
        \cos{ \theta}   &  0            & -\sin{\theta} \\
        0               &  1            &  0            \\
        \sin{ \theta}   &  0            &  \cos{\theta}
    \end{bmatrix}
    \begin{bmatrix}
        \cos{\psi}      &  \sin{\psi}   &  0             \\
       -\sin{\psi}      &  \cos{\psi}   &  0             \\
        0               &  0            &  1
    \end{bmatrix}
\end{align*}
\begin{align} \tag{1.3-10}
    C_{f\!r\!d\!/\!n\!e\!d} &=
    \begin{bmatrix}
        c\theta c\psi   & c\theta s\psi & -s\theta    \\
        \left(-c\phi s\psi + s\phi s\theta c\psi \right) 
        &  \left( c\phi c\psi + s\phi s\theta s\psi \right) 
        &  s\phi c\theta                                 \\
        \left( s\phi s\psi + c\phi s\theta c\psi \right) 
        &  \left( -s\phi c\psi + c\phi s\theta s\psi \right) 
        & c\phi c\theta
    \end{bmatrix}
\end{align}

This matrix represents a standard transformation and will be used
throughout the text.

The defined ranges for the rotation angles are
\begin{align*}
    -\pi  < \phi \leq \pi \\
    -\frac{\pi}{2} \leq \theta \leq \frac{\pi}{2} \\
    -\pi < \psi \leq \pi
\end{align*}

If the pitch angle, \( \theta \), had beem allowed to have a \( \pm
180^{\circ} \) range then the airplane could be inverted and
heading South with the roll and heading angles reading zero, which
is obviously undesirable from a human factor viewpoint! The
restriction on theta can be enforced naturally, simply by
interpretation of the DCM, as we see in the next subsection.

\section{Cinemática e Ângulos de Euler}
\\

\textbf{ADAPTADO de \citeonline{Stevens2016}.}

\\
Para vincular as taxas de ângulos de Euler, que descrevem a mudança de atitude
de um corpo, à sua velocidade angular, procedemos do seguinte modo. Definimos
um quadro de referência \(F_{r}\) e um quadro do corpo \(F_{b}\) com vetor
velocidade angular relativa \(\omega_{b/r}\) e uma sequencia de ângulos de Euler
que define a atitude do corpo, ou seja, a orientação do sistema de coordenadas preso
ao corpo em relação ao sistema de referência. Cada taxa de ângulos de Euler
informa a direção e magnitude para um determinado vetor velocidade angular
sobre um eixo de coordenadas em particular. Esses três vetores somados formam o
vetor velocidade angular resultante do veículo cujas taxas de ângulos de Euler
estamos tratando. Desse modo podemos encontrar os componentes do vetor
velocidade angular resultante.

Em outras palavras, estamos tratando de movimento sobre a Terra, com um sistema de
coordenada \(frd\) (\emph{``front'', ``right'', ``down''} - frente, direita,
abaixo) preso no corpo, com o sistema \(ned\) (\emph{``north'', ``east'', ``down''}
- norte, leste abaixo) no quadro de referência, e uma sequencia
``yaw-pitch-roll'' de ângulos de Euler do sistema \(ned\) para o sistema
\(frd\). No caso das equações de Terra plana o sistema \(ned\) é fixado na
Terra, e a velocidade angular relativa é aquela do corpo em relação à Terra. No
caso mais geral das equações com seis graus de liberdade o sistema \(ned\) se
move sobre a Terra, abaixo do corpo, e devemos estabelecer um quadro de
referência abstrato que tem sua própria velocidade angular em relação ao
sistema da Terra (determinada pela latitude e longitude).

As transformações de coordenadas são:
\begin{align*}
    \mathbf{\omega}^{frd}_{b/r} = \begin{bmatrix} \dot\phi \\ 0 \\0 \end{bmatrix}
    + C_{\phi} \begin{pmatrix}
        \begin{bmatrix} 0 \\ \dot\theta \\ 0 \end{bmatrix}
        + C_{\theta}\begin{bmatrix} 0 \\ 0 \\ \dot\psi \end{bmatrix}
    \end{pmatrix}
\end{align*}

\ldots sendo \(C_{\phi}\) e \(C_{\theta}\)  as rotações (anti-horárias) dos
planos por cada ângulo de Euler em particular, conforme equação (1.3-10). Após
multiplicar as matrizes, teremos:

\begin{align} \tag{1.4-3}
    \mathbf{\omega}^{frd}_{b/r} \equiv \begin{bmatrix} P \\ Q \\ R \end{bmatrix}
    = \begin{bmatrix}
        1 & 0 & -\sin{\theta} \\
        0 & \cos{\phi} & \sin{\phi}\cos{\theta} \\
        0 & -\sin{\phi} & \cos{\phi}\cos{\theta}
    \end{bmatrix}
    \begin{bmatrix}
        \dot\phi \\
        \dot\theta \\
        \dot\psi
    \end{bmatrix}
\end{align}

\ldots onde \(P\), \(Q\), \(R\), são os símbolos padrão para os componentes do
vetor velocidade angular do corpo espresso no sistema \(frd\), respectivamente,
rolagem (\emph{``roll''}), arfada (\emph{``pitch''}) e guinada
(\emph{``yaw''}). A transformação inversa é dada por:

\begin{align} \tag{1.4-4}
    \begin{bmatrix}
        \dot\phi \\
        \dot\theta \\
        \dot\psi
    \end{bmatrix}
    =
    \begin{bmatrix}
        1 & \sin{\phi}\tan{\theta} & \cos{\phi}\tan{\theta} \\
        0 & \cos{\phi} & -\sin{\phi} \\
        0 & \frac{\sin{\phi}}{\cos{\theta}} & \frac{\cos{\phi}}{\cos{\theta}}
    \end{bmatrix}
    \begin{bmatrix}
        P \\ Q \\ R
    \end{bmatrix}
\end{align}

Para simplificar, definimos \(\Phi \equiv \left[\phi \theta \psi \right]^T \) e reescrevemos  (1.4-4) assim:

\begin{equation} \tag{1.4-5}
    \Phi = H \left( \Phi \right) \mathbf{\omega}^{frd}_{b/r}
\end{equation}

As equações (1.4-3) e (1.4-4) serão referidos como as equações cinemáticas de
Euler. Note que as matrizes de coeficientes \emph{não são} matrizes ortogonais
representando rotações ordinárias de coordenadas. Note ainda que as Equações
(1.4-4) e (1.4-4) possuem uma singularidade quando  \(\theta = \pm
\frac{\pi}{2}\). Ainda, se essas equações forem utilizadas em uma simulação, as
taxas de ângulos de Euler podem integrar para ângulos fora do intervalos de
ângulos de Euler, e portanto, seria necessário incluir uma lógica para lidar
com essa situação no programa simulador. Não obstante, as equações cinemáticas
de Euler são bastante empregadas em simulações.

