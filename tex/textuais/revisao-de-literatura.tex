\chapter{REVISÃO DE LITERATURA}\label{chap:fundamentacaoTeorica}

Fazer referências às fontes consultadas:
\\

\textbf{Manuais de Produto}:

Especificação do Produto~\cite{mpu6050ps},

Descrição e Mapa de Registradores~\cite{mpu6050rm}.

\textbf{Controle e Robótica}:

Sistemas de Controle Moderno~\cite{Ogata2010},

Introdução à Robótica~\cite{Craig2014},

\textbf{Ângulos de Euler}:

Ângulos de Euler~\cite{Henderson1997},

Mecânica Clássica~\cite{Goldstein1980},

\textbf{Sistemas de Navegação Inercial}:

Navegação Inercial Básica~\cite{Stovall1997}.

Navegação Inercial Fixada no Corpo~\cite{Weston2004},

Sistemas de Controle de Aeronaves~\cite{Stevens2016},

Controle Automático de Aeronaves e Mísseis~\cite{Blakelock1991},

Navegação Inercial Pedestre~\cite{Wang2021},

Sistemas de Navegação Embarcados~\cite{Haoran2019}

\section{Notação e convenções}

Neste trabalho utilizamos um modelo de mundo tridimensional e mecânica clássica.  Nosso espaço é estruturado por três eixos ortogonais \(x\), \(y\), \(z\) (\(1, 2, 3\)) onde uma posição é dada por um vetor tridimensional. Descreveremos a atitude e o movimento de um corpo rígido sobre a superfície oblata e girante da Terra. Não obstante, serão empregadas aproximações sobre uma pequena área para uma terra plana, estacionária e constante, que basta\footnotemark{} para os nossos propósitos.

Empregaremos as convenções em~\cite{Stevens2016}:
\begin{align*}
    \mathbf{p}_{A/B} &\equiv
    \text{ vetor posição do ponto A em relação ao ponto B} \\
    \mathbf{v}_{A/i} &\equiv
    \text{ vetor velocidade do ponto \(A\) no quadro \(F_{i}\)} \\
    ^{b}\mathbf{\dot{v}}_{A/i} &\equiv
    \text{ vetor derivada de \(\mathbf{v}_{A/i}\) tomada no sistema \( F_{b}\)} \\
    \mathbf{v}^{c}_{A/i} &\equiv \left( \mathbf{v}_{A/i} \right)^{c} \equiv
    \text{ conjunto dos componentes de \(\mathbf{v}_{A/i}\) no sistema de coordenadas \(c\)} \\
\end{align*}

Componentes de vetores terão índices para indicar o sistema de coordenadas ou serão denotados pelo símbolo de vetor com índices \(x\), \(y\), e \(z\) para indicar as coordenadas, exceto quando indicado pelo símbolo transposto, e todos vetores são do tipo vetor coluna. Por exemplo:
\begin{align*}
    &\mathbf{p}^{b}_{A/B} = \begin{bmatrix}x_{b} \\ y_{b} \\ z_{b}\end{bmatrix}&
    \text{ ou }&
    &\mathbf{v}^{b}_{A/i} = \begin{bmatrix} v_{x} \\ v_{y} \\ v_{z} \end{bmatrix} = \begin{bmatrix} v_{x} & v_{y} & v_{z} \end{bmatrix}^{T}
\end{align*}

\footnotetext{A título de advertência, o assunto não é simples, com diversos formalismos possíveis, um campo fértil para confusão. Por exemplo, a atitude de um corpo pode ser descrita em três dimensões com ângulos de Euler ao menos em doze sequências de rotações anti-horárias distintas, não intercambiáveis, ou ainda pode ser descrita em quatro dimensões com o uso do quaternion, conforme observamos em~\cite{Henderson1997}. Além disso, é importante separar a orientação relativa do nosso espaço tridimensional em abstrato, do eixo de referência em relação ao nosso espaço abstrato, e do sistema móvel que pretendemos descrever em relação ao sistema de referência. No presente trabalho adotaremos um único formalismo que atende aos nossos propósitos limitados, e remetemos o leitor às fontes para aprofundamento do assunto.}

\section{Descrevendo a atitude de um corpo}

\textbf{ADAPTADO de \citeonline{Stevens2016}.}

Descrevemos a atitude de um veículo em ângulos de Euler na sequência \(z\), \(y\), \(x\) (3, 2, 1) que leva de do sistema de referência na Terra até um sistema alinhado com o corpo do veículo. É comum escolher o sistema (\(ned\)) - ``North-East-Down'' (Norte, Leste, Abaixo) com o eixo \(x\) apontando par o Norte, o eixo \(z\) Abaixo, o eixo \(y\) completando o sistema de coordenadas. Os eixos do sistema (\(frd\)) - ``Front-Right-Down'' (Avante, Direita, Abaixo) no veículo são, respectivamente, (\(x\),\(y\),\(z\)), sendo o Avante alinhado à \emph{linha de referência longitudinal} do veículo, com os eixos  ``Avante'' e ``Abaixo'' situados no plano de simetria do veículo. Desse a modo, a sequência de rotação que leva do sistema de referência \(ned\) para o sistema \(frd\) no corpo é dada por:

\begin{enumerate}
    \item Rotação anti-horária sobre eixo \(z\), ou \(\psi\) positivo (\textit{``compass heading''})
    \item Rotação anti-horária sobre novo eixo \(y'\), ou \(\theta\) positivo (\textit{pitch})
    \item Rotação anti-horária sobre novo eixo \(x''\), ou \(\phi\) positivo (\textit{roll})
\end{enumerate}

Esta sequência de rotações é normalmente denominada \emph{``yaw, pitch, roll''}, partindo do sistema de referência.

Podemos escrever as matriz de rotação (abreviando co-seno por \(c\) e seno por \(s\)):

\begin{align*}
    C_{f\!r\!d\!/\!n\!e\!d} =
    \begin{bmatrix}
        1               &  0            &  0             \\
        0               &  \cos{\phi}   &  \sin{\phi}    \\
        0               & -\sin{\phi}   &  \cos{\phi}
    \end{bmatrix}
    \begin{bmatrix}
        \cos{ \theta}   &  0            & -\sin{\theta} \\
        0               &  1            &  0            \\
        \sin{ \theta}   &  0            &  \cos{\theta}
    \end{bmatrix}
    \begin{bmatrix}
        \cos{\psi}      &  \sin{\psi}   &  0             \\
       -\sin{\psi}      &  \cos{\psi}   &  0             \\
        0               &  0            &  1
    \end{bmatrix}
\end{align*}
\begin{align} \tag{1.3-10}
    C_{f\!r\!d\!/\!n\!e\!d} &=
    \begin{bmatrix}
        c\theta c\psi   & c\theta s\psi & -s\theta    \\
        \left(-c\phi s\psi + s\phi s\theta c\psi \right) 
        &  \left( c\phi c\psi + s\phi s\theta s\psi \right) 
        &  s\phi c\theta                                 \\
        \left( s\phi s\psi + c\phi s\theta c\psi \right) 
        &  \left( -s\phi c\psi + c\phi s\theta s\psi \right) 
        & c\phi c\theta
    \end{bmatrix}
\end{align}

Esta matriz representa uma transformação padrão que será utilizada ao longo do texto.

O intervalo de validade para o qual os ângulos de rotação são bem definidos é:
\begin{align*}
    -\pi  < \phi \leq \pi \\
    -\frac{\pi}{2} \leq \theta \leq \frac{\pi}{2} \\
    -\pi < \psi \leq \pi
\end{align*}

Caso o ângulo \( \theta \) fosse definido no intervalo de \( \pm 180^{\circ} \) o veículo estaria apontando para o Sul com os ângulos \(\phi\) e \(\psi\) em \( 0^{\circ}\) o que é indesejável pois pode confundir a interpretação.


\section{Cinemática Rotacional}

Aqui definiremos a derivada de um vetor, mostraremos como ela depende do sistema de referência do observador, e relacionamos as derivadas de um vetor, tomadas em dois sistemas de referência distintos, através do vetor velocidade angular relativa entre esses dois sistemas.

Genericamente a derivada de um vetor é similar à derivada de um escalar:
\begin{equation*}
    \frac{\mathrm{d}}{\mathrm{d}t} \mathbf{p}_{A/B} =  \lim_{\delta t \rightarrow 0 } \begin{bmatrix}
        \displaystyle\frac{\mathbf{p}_{A/B} (t + \delta t) - \mathbf{p}_{A/B} (t) }{\delta t}
    \end{bmatrix}
\end{equation*}

Este é novo vetor criado pelas mudanças de comprimento e direção de \(\mathbf{p}_{A/B}\). Sendo \(\mathbf{p}\) um vetor livre, por exemplo, a velocidade, esperamos que sua derivada seja independente de translação, e que as mudanças de comprimento e direção decorram do movimento da ponta de \(\mathbf{p}\) em relação à cauda. Se \(\mathbf{p}\) seja um vetor vinculado a algum sistema, por exemplo o vetor posição, sua derivada naquele sistema é um vetor livre que corresponde à ponta de \(\mathbf{p}\).

\subsection{Velocidade Angular como Vetor}

Um vetor pode apontar em qualquer direção por meio de uma simples rotação ao longo de um eixo apropriado. A fórmula para essa rotação é descrita em~\cite{Goldstein1980}:

\begin{figure}[H]
    \centering
    \includegraphics[width=0.5\textwidth, keepaspectratio]{figuras/figure1.2-1.png}\label{fig1.2-1}
    \caption{Rotação de um vetor}
\end{figure}

Na figura acima, o vetor \(\mathbf{u}\) foi rotacionado para formar o vetor \(\mathbf{v}\) ao definirmos um eixo de rotação ao longo do vetor \(\mathbf{n}\) e realizarmos uma rotação pelo ângulo \(\mu\) ao redor de \(\mathbf{n}\). Estes dois vetores se somam a \(\mathbf{u}\) para obtermos \(\mathbf{v}\), e onde temos que:
    \begin{align*}
     \mathbf{v} &= \mathbf{u} + \left(1 - {\cos{\mu}}\right) \left(\mathbf{n}\!\times\!\left(\mathbf{n}\!\times\!\mathbf{u}\right)\right) - \left(\mathbf{n}\!\times\!\mathbf{u}\right){\sin{\mu}} \tag{1.2-5a}
    \end{align*}
    \ldots ou
    \begin{align*}
     \mathbf{v} = \left(1 - {\cos{\mu}}\right) \mathbf{n}\!\left(\mathbf{n}\cdot\mathbf{u}\right) + \mathbf{u}{\cos{\mu}} - \left(\mathbf{n}\!\times\!\mathbf{u}\right){\sin{\mu}} \tag{1.2-5b}
    \end{align*}

    As equações acima (1.2-5), às vezes chamadas de \textit{formula de rotação}, mostram ao definirmos \(\mathbf{n}\) e \(\mu\) podemos operar sobre \(\mathbf{u}\) com produtos escalares e vetoriais para obter a rotação deseja, independente de sistemas de coordenadas ou magnitude do ângulo.

    Partindo da figura acima, fazemos uma rotação infinitesimal \(\delta\mu \ll 1 \text{rad}\), definindo \(\mathbf{v} = \mathbf{u} + \delta \mathbf{u}\), e a partir da equação (1.2-5\(a\)) obtemos:
\begin{equation*}
    \delta \mathbf{u} \approx -\!\sin(-\delta\mu)\mathbf{n}\!\times\!\mathbf{u} \approx (\mathbf{n}\!\times\!\mathbf{u})\delta\mu
\end{equation*}

Dividindo por \(\delta t\), no limite em que \(\delta t \rightarrow 0\), definimos \(\mathbf{\omega} \equiv \dot{\mu}\mathbf{n}\), obtendo:

\begin{equation*}
    \dot{\mathbf{u}} = \mathbf{\omega}\!\times\!\mathbf{u}\tag{1.4-1}
\end{equation*}

Esta equação relaciona a velocidade translacional da ponta do vetor, vinculado e de comprimento constante, \(\mathbf{u}\) ao vetor \(\mathbf{\omega}\). O vetor \(\mathbf{\omega}\) constitui-se de um vetor unitário definindo o eixo de rotação, multiplicado pela taxa de rotação, e o denominamos \textit{vetor velocidade angular} dessa rotação. Este é um vetor livre que pode ser trasladado paralelo a si mesmo e um pseudovetor axial, que mudaria de direção caso houvéssemos escolhido uma convenção de sentido horário para a rotação.

Sendo \(\mathbf{\omega}\) um vetor livre, podemos associar ao sistema fixado no corpo, atribuindo índices que indicam que ele representa a velocidade angular do corpo em relação a outro determinado sistema. A orientação de um corpo rígido pode ser descrita por uma matriz rotacional variante no tempo, e por consectário do teorema de Euler o corpo possui um único \emph{eixo instantâneo de rotação} ao qual o velocidade angular é paralelo e também único.

\section{Cinemática e Ângulos de Euler}

\textbf{ADAPTADO de \citeonline{Stevens2016}.}
\\

Um corpo em movimento pode mudar sua atitude ao longo do tempo, descrita em ângulos de Euler que vão mudando ao longo do tempo, e neste sentido podemos falar de uma taxa de mudança de cada um desses ângulos de Euler ao longo do tempo. Essas taxas são coisa distinta, é preciso dizer, do vetor velocidade angular do corpo.

Para vincular as taxas de ângulos de Euler, que descrevem a mudança de atitude de um corpo, à sua velocidade angular, procedemos do seguinte modo. Definimos um quadro de referência \(F_{r}\) e um quadro do corpo \(F_{b}\) com vetor velocidade angular relativa \(\omega_{b/r}\) e uma sequencia de ângulos de Euler que define a atitude do corpo, ou seja, a orientação do sistema de coordenadas preso ao corpo em relação ao sistema de referência. Cada taxa de ângulos de Euler informa a direção e magnitude para um determinado vetor velocidade angular sobre um eixo de coordenadas em particular. Esses três vetores somados formam o vetor velocidade angular resultante do veículo cujas taxas de ângulos de Euler estamos tratando. Desse modo podemos encontrar os componentes do vetor velocidade angular resultante.

Em outras palavras, estamos tratando de movimento sobre a Terra, com um sistema de coordenada \(frd\) (\emph{``front'', ``right'', ``down''} - frente, direita, abaixo) preso no corpo, com o sistema \(ned\) (\emph{``north'', ``east'', ``down''} - norte, leste abaixo) no quadro de referência, e uma sequencia ``yaw-pitch-roll'' de ângulos de Euler do sistema \(ned\) para o sistema \(frd\). No caso das equações de Terra plana o sistema \(ned\) é fixado na Terra, e a velocidade angular relativa é aquela do corpo em relação à Terra. No caso mais geral das equações com seis graus de liberdade o sistema \(ned\) se move sobre a Terra, abaixo do corpo, e devemos estabelecer um quadro de referência abstrato que tem sua própria velocidade angular em relação ao sistema da Terra (determinada pela latitude e longitude).

As transformações de coordenadas são:

\begin{align*}
    \mathbf{\omega}^{frd}_{b/r} = \begin{bmatrix} \dot\phi \\ 0 \\0 \end{bmatrix}
    + C_{\phi} \begin{pmatrix}
        \begin{bmatrix} 0 \\ \dot\theta \\ 0 \end{bmatrix}
        + C_{\theta}\begin{bmatrix} 0 \\ 0 \\ \dot\psi \end{bmatrix}
    \end{pmatrix}
\end{align*}

\ldots sendo \(C_{\phi}\) e \(C_{\theta}\)  as rotações (anti-horárias) dos
planos por cada ângulo de Euler em particular, conforme equação (1.3-10). Após
multiplicar as matrizes, teremos:

\begin{align} \tag{1.4-3}
    \mathbf{\omega}^{frd}_{b/r} \equiv \begin{bmatrix} P \\ Q \\ R \end{bmatrix}
    = \begin{bmatrix}
        1 & 0 & -\sin{\theta} \\
        0 & \cos{\phi} & \sin{\phi}\cos{\theta} \\
        0 & -\sin{\phi} & \cos{\phi}\cos{\theta}
    \end{bmatrix}
    \begin{bmatrix}
        \dot\phi \\
        \dot\theta \\
        \dot\psi
    \end{bmatrix}
\end{align}

\ldots sendo \(P\), \(Q\), \(R\), os componentes do vetor velocidade angular do corpo espresso no sistema \(frd\), respectivamente, rolagem (\emph{``roll''}), arfada (\emph{``pitch''}) e guinada (\emph{``yaw''}). A transformação inversa é dada por:

\begin{align} \tag{1.4-4}
    \begin{bmatrix}
        \dot\phi \\
        \dot\theta \\
        \dot\psi
    \end{bmatrix}
    =
    \begin{bmatrix}
        1 & \sin{\phi}\tan{\theta} & \cos{\phi}\tan{\theta} \\
        0 & \cos{\phi} & -\sin{\phi} \\
        0 & \frac{\sin{\phi}}{\cos{\theta}} & \frac{\cos{\phi}}{\cos{\theta}}
    \end{bmatrix}
    \begin{bmatrix}
        P \\ Q \\ R
    \end{bmatrix}
\end{align}

Para simplificar, definimos \(\Phi \equiv \left[\phi \theta \psi \right]^T \) e reescrevemos  (1.4-4) assim:

\begin{equation} \tag{1.4-5}
    \Phi = H \left( \Phi \right) \mathbf{\omega}^{frd}_{b/r}
\end{equation}

As equações (1.4-3) e (1.4-4) serão referidos como as equações cinemáticas de Euler. Note que as matrizes de coeficientes \emph{não são} matrizes ortogonais representando rotações ordinárias de coordenadas. Note ainda que as Equações (1.4-4) e (1.4-4) possuem uma singularidade quando  \(\theta = \pm \frac{\pi}{2}\). Ainda, se essas equações forem utilizadas em uma simulação, as taxas de ângulos de Euler podem integrar para ângulos fora do intervalos de ângulos de Euler, e portanto, seria necessário incluir uma lógica para lidar com essa situação no programa simulador. Não obstante, as equações cinemáticas de Euler são bastante empregadas em simulações.

