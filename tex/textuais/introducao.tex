\chapter{INTRODUÇÃO}
\label{chap:introducao}

Sistemas robóticos, em geral, são empregados quando a intervenção humana
revelas-e muito onerosa, perigosa ou inefixas. A capacidade de operar de forma
autônoma, nestes casos, é uma caracterísitca valiosa, viabilizada através de
sistemas para monitorar e controlar o movimento nos deslocamentos entre um
ponto e outro. Para elevar o nível de precisão desse controle os sensores de
medição inercial oferecem um mecanismo de retroalimentação dos sistemas de
controle que pode ser muito útil na otimização desses sistemas de navegação e
controle. O presente trabalho pretende avaliar a utilidade de um particular
modelo de sensor inercial de baixo custo em sistemas robóticos dessa natureza
mediante a criação de um driver para este sensor ligado a uma Raspberry Pi.
