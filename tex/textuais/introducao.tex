\chapter*{INTRODUÇÃO}
\addcontentsline{toc}{chapter}{INTRODUÇÃO}

Sistemas robóticos, em geral, são empregados quando a intervenção humana
revela-se muito onerosa, perigosa ou ineficaz. A capacidade de operar de forma
autônoma, nestes casos, é uma caracterísitca valiosa, viabilizada através de
sistemas para monitorar e controlar o movimento nos deslocamentos entre um
ponto e outro.

Esses sistemas de controle e monitoramento devem apresentar respostas
satisfatórias para questões que abrangem, por exemplo, as coordenadas de
posicionamento global, a orientação espacial em relação ao plano tangente à
terra, a velocidade e aceleração do robô no espaço, distância de obstáculos
próximos, dentre outras. Embora para um ser humano essas respostas possam
parecer intuitivas, num primeiro momento, o sistema robótico depende
exclusivamente dos seus sensores e algoritmos para respondê-las.

Dentre os sensores mais comuns nesses sistemas, temos os hodômetros
implementados através de \emph{encoders}, que medem o deslocamento
proporcionado pelas rodas ou esteiras do robô, os sensores de distância
ultrasônicos, direcionados ao ambiente externo para obter eco em objetos
próximos do robô, sensores de medição de distância a laser, também conhecido
por \emph{LIDAR}, que permitem obter ecos de corpos relativamente mais
distantes do robô, os sensores de posição baseados em GPS, os magnetômetros que
funcionam como um bússola, os sensores de medição inercial como acelerômetros e
giroscópios, e hoje até mesmo sensores de imagem como câmeras. E embora sejam
objeto de um sofisticado tratamento matemático, nenhum desses sensores,
isoladamente, satifaz a demanda por precisão das respostas que se espera num
sistema de controle e monitoramento.

Para elevar o nível de precisão, os sensores de medição inercial oferecem um
mecanismo de retroalimentação que, aparentemente, pode ser muito útil na
otimização dos sistemas de controlerobótibo. O presente trabalho pretende
avaliar a utilidade de um particular modelo de sensor inercial de baixo custo
em sistemas robóticos dessa natureza mediante a criação de um driver para este
sensor ligado a uma Raspberry Pi.

