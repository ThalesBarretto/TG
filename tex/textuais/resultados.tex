\chapter{RESULTADOS}\label{chap:resuldatos}

A seguir são apresentados os resultados obtidos a partir dos testes de bancada realizados. Os resultados serão apresentados contrapondo os dados obtidos contra os dados esperados.

\section{Resultado do teste estacionário}

Inicialmente realizamos o teste mais simples possível: deixamos o sensor estacionário em uma bancada nivelada e estável, sem vibrações significativas ou perturbações. O teste foi realizado três vezes consecutivas, partindo da inicialização do sensor, calibração, e capturas com duração de cinco minutos cada. O resultado esperado, portanto, é uma indicação de zero para todos os valores de atitude e posição ao final do teste. As condições do teste são ilustradas na figura~\ref{fig:mpu6050-proto-top}.

Nestas condições configuramos o sensor para capturar a \(250\) amostras por segundo, integrando em \(200\) passos, configurando um fator de escala para a aceleração da gravidade no valor padrão de \(9.8\textrm{m}/{s}^{2}\), ajustando a sensibilidade dos sensores para \(\pm 4\textrm{m}/{s}^{2}\) e \(\pm 500\textrm{dps}\), sem filtro digital.

Ao final obtivemos estimativas de atitude \((\phi,\theta,\psi)\) e posição~\((p_{n},p_{e},p_{u})\) a seguir:

\begin{table}[ht]
    \caption{Resultados do teste estacionário}
    \centering
    \begin{tabular}{r r r r r r r}
        teste & \(\phi (^{o})\) & \(\theta(^{o})\) & \(\psi(^{o})\) & \(p_{n}(\textrm{m})\) & \(p_{e}(\textrm{m})\) & \(p_{u}(\textrm{m})\)  \\
        \toprule
        1 & +0.09 & +0.05 & +0.05 & -0.308 & -1.174 & +127.510  \\
        2 & +0.09 & -0.01 & -0.01 & +0.396 & -1.458 & +100.242  \\
        3 & +0.06 & -0.02 & -0.02 & -0.647 & -0.636 & +105.201  \\
        \bottomrule
    \end{tabular}
    \label{Tab:tabela1}
\end{table}

Como podemos observar, nos três testes nossa estimativa de posição \(p_{u}\) ficou mais de cem metros distante do ponto inicial, embora as demais estimativas estivessem razoavelmente próximas dos resultados esperados.

Em razão da significativa divergência entre os resultados obtidos e os esperados, decidimos não realizar os testes cinemáticos, por considerá-los dispensáveis neste particular.
