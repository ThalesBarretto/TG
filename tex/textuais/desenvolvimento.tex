\chapter{DESENVOLVIMENTO}\label{chap:desenvolvimento}

Para operar os sensores inerciais, consultamos extensamente a documentação do fabricante: Especificação do Produto~\cite{mpu6050ps} e Descrição e Mapa de Registradores~\cite{mpu6050rm}.
A partir deles, estabelecemos como requisitos do sistema:
\begin{enumerate}
        \item obter dados simultâneos de todos os sensores em cada amostra,
        \item obter dados em intervalos regulares de tempo,
        \item obter dados em grandezas com significado físico,
        \item permitir o controle da sensibilidade dos sensores
        \item permitir o controle da taxa de amostragem
        \item permitir o controle das funções de filtro
        \item seja utilizável com sistemas operacionais livres e amplamente disponíveis
        \item possa ser instalado como um programa sem modificar o sistema operacional
        \item possa ser livremente distribuído com licença de código aberto
        \item possa ser operado pela linha de comando
\end{enumerate}

Entretanto, não encontramos nenhuma solução pronta capaz de atender aos nossos requisitos. Nos restou, então, a partir dos manuais, desenvolver programas para controlar o sensor e obter os dados.

Os programas de nossa autoria foram escritos em linguagem C para sistemas Linux, rodam em uma \emph{Raspberry Pi 3B}, estão disponíveis\footnote{O driver em \href{https://www.github.com/ThalesBarretto/libmpu6050}{https://www.github.com/ThalesBarretto/libmpu6050}}\footnote{A biblioteca auxiliar em \href{https://www.github.com/ThalesBarretto/libmtx}{https://www.github.com/ThalesBarretto/libmtx}}\footnote{A aplicação em \href{https://www.github.com/ThalesBarretto/mpu6050}{https://www.github.com/ThalesBarretto/mpu6050}}sob licença permissiva de código aberto\footnote{Sob a ``MIT License'' disponível em \href{https://mit-license.org/}{https://mit-license.org/} e \href{https://opensource.org/licenses/MIT}{https://opensource.org/licenses/MIT}}, e acreditamos atender aos requisitos.

Foram realizados testes de bancada estáticos \ldots

Os resultados foram registrados no formato \ldots

\section{Ângulos a partir do acelerômetro}

Nesta seção apresentaremos como estimar orientação a partir da leitura dos acelerômetros.

\section{Ângulos a partir dos giroscópios}

Nesta seção apresentaremos como estimar orientação a partir da leitura dos giroscópios.

\section{Fusão de sensores}

Nesta seção apresentaremos como combinar as estimativas de acelerômetro e giroscópio.
