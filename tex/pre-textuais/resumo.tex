\begin{resumo}[RESUMO]
\begin{SingleSpacing}

% Não altere esta seção do
% texto--------------------------------------------------------
%\imprimirautorcitacao. \imprimirtitulo. \imprimirdata. \pageref {LastPage} f.
%\imprimirprojeto\ – \imprimirprograma, \imprimirinstituicao. \imprimirlocal,
%\imprimirdata.\\
%---------------------------------------------------------------------------------------

    O objetivo deste trabalho foi desenvolver um sistema, o mais simples possível, para captura de atitude e movimento para sistemas robóticos autônomos empregando um sensor inercial de baixo custo.
    Empregamos a abordagem matemática mais simples disponível na literatura para desenvolver duas peças de software capazes de implementar as equações.
    Desenvolvemos um driver e uma aplicação de demonstração que, respectivamente, controlam o sensor MPU6050 e implementam as equações cinemáticas de Euler para capturar atitude e posição.
    Realizamos testes de bancada para a obtenção de dados que revelem a utilidade e adequação dos métodos e materiais empregados para a finalidade proposta.
    Ao final, discutimos os resultados obtidos e as limitações do materiais e métodos, indicando abordagens diversas que podem oferecer resultados melhores.

    \textbf{Palavras-chave}: Robótica. Sensores de Medição Inercial. Sistemas
    robóticos autônomos.

\end{SingleSpacing}
\end{resumo}

